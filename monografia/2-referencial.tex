\newcommand{\chisquare}{$\chi^2$}

\chapter{Referêncial Teórico}

\section{Esteganografia}
%-----------------------

Esteganografia, em um contexto atual, é a prática de esconder um aquivo digital dentro de outro. Estes arquivos podem ser imagens, vídeos, aúdio, ou simplesmente texto. A palavra estaganografia combina as palavras gregas \emph{steganos}, que significa ``coberto'', ou ``protegido'', e \emph{graphein}, que significa ``escrita''.

Apesar do uso moderno comumente envolver arquivos de computador, a esteganografia é uma prática antiga que tem muitos anos de história.

A esteganografia contrasta com a criptografia quanto ao aspecto da confidencialidade. A criptografia oferece um meio de comunicação seguro, onde terceiros não podem enteneer o que está sendo conversado, porém sabe que informação confidencial está sendo trocada.

O objetivo da esteganografia é modificar o arquivo portador de forma imperceptível, de maneira que nada seja revelado. Nem a presença de uma mensagem secreta e, muito menos, a mensagem secreta em si. \cite{westfeld1999attacks}

Sistemas esteganográficos geralmente funcionam processando dois parâmetros: uma mensagem secreta, e uma mensagem de cobertura. \cite{westfeld1999attacks}  % Tem uma imagem bonitinha no artigo que pode ser útil.
Alguns também precisam de uma senha para criptografar a mensagem. \cite{??}

Arquivos de múltimídia, como aúdio e vídeo, servem como exceletentes portadores. Pois contém rúido que serve de espaço para inserir uma mensagem secreta. \cite{westfeld1999attacks}

Um esteganograma deve ter as mesmas características estatísticas do aquivo original. De forma contrária, o sistema esteganográfico seria inseguro. \cite{westfeld1999attacks} Pois a presença de uma mensagem secreta seria mais facilmente detectada.

No uso moderno existem dois tipo principais de esteganografia: \emph{Least Significant Bit} (LSB) e \emph{Discrete Cosine Transform} (DCT).


\subsubsection{Least Significant Bit (LSB)}

\emph{Least Significant Bit} (LSB) (literalmente, bit menos significante) é o método mais básico de estaganografia moderna. O bit menos significativo é o que torna uma sequência de bits par ou impar. No byte \texttt{0110110\textcolor{red}{1}}, o bit menos significativo é o bit \texttt{\textcolor{red}{1}} que está na extrema esquerda.

Ao alterar o LSB o valor do byte não altera muito, se comprado a mudança a qualquer um dos outros bits. E é por isso que muitos métodos de esteganografia alteram apenas os LSBs, assim o valor de cada byte não terá uma alteração significativa. Em uma imagem, por exemplo, após aplicar o método LSB para esconder uma mensagem, os bits mais a direita terão sido alterados, mas a imagem, para o olho humano, continua a mesma. Não há mudanças perceptíveis para o olho humano.

Alterando os LSBs, a capacidade máxima que temos para esconder uma informação é de $1/8$ do tamanho do arquivo. Se uma imagem tem 152 Kilobytes, podemos esconder 19 Kilobytes (152 Kilobits) de informação.


\subsubsection{Discrete Cosine Transform (DCT)}

No formato de imagem JPEG, para cada componente de cor o formato usa \emph{Discrete Cosine Transform} (DCT) para transformar blocos de 8x8 pixels em coeficentes 64 DCTs cada. \cite{provos_hide_2003}


\section{Esteganálise}
%---------------------

A esteganálise é uma área de pesquisa que busca criar métodos para detectar quando a esteganografia foi aplicada em um item ou não.

Existem dois tipos principais de esteganálise: visuais e estatísticos. Quandos disponíveis, ataques estatísticos são superiores aos visuais, pois são menos dependentes do arquivo original e podem ser completamente automatizados, o quê permite processar vários itens em larga escala. \cite{westfeld1999attacks}

Alguns dos principais métodos de esteganálise são: Chi Square, RS Analysis, Primary Sets, Sample Pair.

\subsection{Chi Square}
O \emph{\chisquare} (Chi Square) \cite{westfeld1999attacks} é uma análise estatística dos pares de valores (PoVs) trocados durante a esteganografia LSB.

Sobrescrever os LSBs de uma imagem transforma alguns valores em outros valores que apenas diferem pelo LSB. Esses pares de valores são chamados de PoV. \texttt{0110110\textcolor{red}{1}} forma um PoV com \texttt{0110110\textcolor{red}{0}}. Se os LSBs sobrescrescritos forem uniformemente distribuídos, as frequências dos valores de cada PoV irá se igualar. Veja a figura fig . % Fig. 14

A ideia do Chi Square é comparar as distribuição de frequências téorica esperada de esteganogramas com algumas amostras da distribuição observada no item que está sendo análisado.

\subsection{RS Analysis}
\cite{fridrich2001reliable}: detecta LSBs espalhados em imagens em tons de cinza ou coloridas ao inspecionar as diferenças no número de grupos regulares e singulares para os LSB e o plano LSB deslocado.

\subsection{Primary Sets}
\cite{dumitrescu2002steganalysis}: baseado em um identidade estatística relacionada a alguns conjuntos de pixels em uma imagem.

\subsection{Sample Pair}
\cite{dumitrescu2003detection}: baseado em uma máquina de estados finitos cujos estados são múltiplos conjuntos selecionados de pares de amostras chamados ``trace multisets''.


\section{Curva ROC}
%------------------


